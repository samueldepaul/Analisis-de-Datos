% Options for packages loaded elsewhere
\PassOptionsToPackage{unicode}{hyperref}
\PassOptionsToPackage{hyphens}{url}
%
\documentclass[
]{article}
\usepackage{amsmath,amssymb}
\usepackage{lmodern}
\usepackage{ifxetex,ifluatex}
\ifnum 0\ifxetex 1\fi\ifluatex 1\fi=0 % if pdftex
  \usepackage[T1]{fontenc}
  \usepackage[utf8]{inputenc}
  \usepackage{textcomp} % provide euro and other symbols
\else % if luatex or xetex
  \usepackage{unicode-math}
  \defaultfontfeatures{Scale=MatchLowercase}
  \defaultfontfeatures[\rmfamily]{Ligatures=TeX,Scale=1}
\fi
% Use upquote if available, for straight quotes in verbatim environments
\IfFileExists{upquote.sty}{\usepackage{upquote}}{}
\IfFileExists{microtype.sty}{% use microtype if available
  \usepackage[]{microtype}
  \UseMicrotypeSet[protrusion]{basicmath} % disable protrusion for tt fonts
}{}
\makeatletter
\@ifundefined{KOMAClassName}{% if non-KOMA class
  \IfFileExists{parskip.sty}{%
    \usepackage{parskip}
  }{% else
    \setlength{\parindent}{0pt}
    \setlength{\parskip}{6pt plus 2pt minus 1pt}}
}{% if KOMA class
  \KOMAoptions{parskip=half}}
\makeatother
\usepackage{xcolor}
\IfFileExists{xurl.sty}{\usepackage{xurl}}{} % add URL line breaks if available
\IfFileExists{bookmark.sty}{\usepackage{bookmark}}{\usepackage{hyperref}}
\hypersetup{
  pdftitle={Ejercicios Tema 1. Primera Entrega.},
  hidelinks,
  pdfcreator={LaTeX via pandoc}}
\urlstyle{same} % disable monospaced font for URLs
\usepackage[margin=1in]{geometry}
\usepackage{color}
\usepackage{fancyvrb}
\newcommand{\VerbBar}{|}
\newcommand{\VERB}{\Verb[commandchars=\\\{\}]}
\DefineVerbatimEnvironment{Highlighting}{Verbatim}{commandchars=\\\{\}}
% Add ',fontsize=\small' for more characters per line
\usepackage{framed}
\definecolor{shadecolor}{RGB}{248,248,248}
\newenvironment{Shaded}{\begin{snugshade}}{\end{snugshade}}
\newcommand{\AlertTok}[1]{\textcolor[rgb]{0.94,0.16,0.16}{#1}}
\newcommand{\AnnotationTok}[1]{\textcolor[rgb]{0.56,0.35,0.01}{\textbf{\textit{#1}}}}
\newcommand{\AttributeTok}[1]{\textcolor[rgb]{0.77,0.63,0.00}{#1}}
\newcommand{\BaseNTok}[1]{\textcolor[rgb]{0.00,0.00,0.81}{#1}}
\newcommand{\BuiltInTok}[1]{#1}
\newcommand{\CharTok}[1]{\textcolor[rgb]{0.31,0.60,0.02}{#1}}
\newcommand{\CommentTok}[1]{\textcolor[rgb]{0.56,0.35,0.01}{\textit{#1}}}
\newcommand{\CommentVarTok}[1]{\textcolor[rgb]{0.56,0.35,0.01}{\textbf{\textit{#1}}}}
\newcommand{\ConstantTok}[1]{\textcolor[rgb]{0.00,0.00,0.00}{#1}}
\newcommand{\ControlFlowTok}[1]{\textcolor[rgb]{0.13,0.29,0.53}{\textbf{#1}}}
\newcommand{\DataTypeTok}[1]{\textcolor[rgb]{0.13,0.29,0.53}{#1}}
\newcommand{\DecValTok}[1]{\textcolor[rgb]{0.00,0.00,0.81}{#1}}
\newcommand{\DocumentationTok}[1]{\textcolor[rgb]{0.56,0.35,0.01}{\textbf{\textit{#1}}}}
\newcommand{\ErrorTok}[1]{\textcolor[rgb]{0.64,0.00,0.00}{\textbf{#1}}}
\newcommand{\ExtensionTok}[1]{#1}
\newcommand{\FloatTok}[1]{\textcolor[rgb]{0.00,0.00,0.81}{#1}}
\newcommand{\FunctionTok}[1]{\textcolor[rgb]{0.00,0.00,0.00}{#1}}
\newcommand{\ImportTok}[1]{#1}
\newcommand{\InformationTok}[1]{\textcolor[rgb]{0.56,0.35,0.01}{\textbf{\textit{#1}}}}
\newcommand{\KeywordTok}[1]{\textcolor[rgb]{0.13,0.29,0.53}{\textbf{#1}}}
\newcommand{\NormalTok}[1]{#1}
\newcommand{\OperatorTok}[1]{\textcolor[rgb]{0.81,0.36,0.00}{\textbf{#1}}}
\newcommand{\OtherTok}[1]{\textcolor[rgb]{0.56,0.35,0.01}{#1}}
\newcommand{\PreprocessorTok}[1]{\textcolor[rgb]{0.56,0.35,0.01}{\textit{#1}}}
\newcommand{\RegionMarkerTok}[1]{#1}
\newcommand{\SpecialCharTok}[1]{\textcolor[rgb]{0.00,0.00,0.00}{#1}}
\newcommand{\SpecialStringTok}[1]{\textcolor[rgb]{0.31,0.60,0.02}{#1}}
\newcommand{\StringTok}[1]{\textcolor[rgb]{0.31,0.60,0.02}{#1}}
\newcommand{\VariableTok}[1]{\textcolor[rgb]{0.00,0.00,0.00}{#1}}
\newcommand{\VerbatimStringTok}[1]{\textcolor[rgb]{0.31,0.60,0.02}{#1}}
\newcommand{\WarningTok}[1]{\textcolor[rgb]{0.56,0.35,0.01}{\textbf{\textit{#1}}}}
\usepackage{graphicx}
\makeatletter
\def\maxwidth{\ifdim\Gin@nat@width>\linewidth\linewidth\else\Gin@nat@width\fi}
\def\maxheight{\ifdim\Gin@nat@height>\textheight\textheight\else\Gin@nat@height\fi}
\makeatother
% Scale images if necessary, so that they will not overflow the page
% margins by default, and it is still possible to overwrite the defaults
% using explicit options in \includegraphics[width, height, ...]{}
\setkeys{Gin}{width=\maxwidth,height=\maxheight,keepaspectratio}
% Set default figure placement to htbp
\makeatletter
\def\fps@figure{htbp}
\makeatother
\setlength{\emergencystretch}{3em} % prevent overfull lines
\providecommand{\tightlist}{%
  \setlength{\itemsep}{0pt}\setlength{\parskip}{0pt}}
\setcounter{secnumdepth}{-\maxdimen} % remove section numbering
\ifluatex
  \usepackage{selnolig}  % disable illegal ligatures
\fi

\title{Ejercicios Tema 1. Primera Entrega.}
\author{}
\date{\vspace{-2.5em}}

\begin{document}
\maketitle

\textbf{\emph{Pau Vives, Harold Cruz, Samuel de Paúl}}

\textbf{\emph{PARTE I}}

\emph{1.Utiliza el operador \$ para acceder a los datos del tamaño de la
población y almacenarlos como el objeto pop. Luego, use la función sort
para redefinir pop para que esté en orden alfabético. Finalmente, usa el
operador {[} para indicar el tamaño de población más pequeño.}

\begin{Shaded}
\begin{Highlighting}[]
\CommentTok{\#Cargamos los datos:}
\CommentTok{\#install.packages("dslabs")}
\FunctionTok{library}\NormalTok{(dslabs)}
\FunctionTok{data}\NormalTok{(murders)}
\NormalTok{pop }\OtherTok{\textless{}{-}}\NormalTok{ murders}\SpecialCharTok{$}\NormalTok{population}

\CommentTok{\#Ordenamos Alfabéticamente: (Supondremos que se refiere a crecientemente)}
\NormalTok{pop2 }\OtherTok{\textless{}{-}} \FunctionTok{sort}\NormalTok{(pop) }

\CommentTok{\#Tamaño de población más pequeño:}
\NormalTok{min\_pop }\OtherTok{\textless{}{-}}\NormalTok{pop2[}\DecValTok{1}\NormalTok{]}
\NormalTok{min\_pop}
\end{Highlighting}
\end{Shaded}

\begin{verbatim}
## [1] 563626
\end{verbatim}

\emph{2.Ahora, en lugar del tamaño de población más pequeño, encuentra
el índice de la entrada con el tamaño de población más pequeño.
Sugerencia: use order en lugar de sort.}

\begin{Shaded}
\begin{Highlighting}[]
\NormalTok{min\_ind\_pop }\OtherTok{=} \FunctionTok{order}\NormalTok{(pop)[}\DecValTok{1}\NormalTok{]}
\NormalTok{min\_ind\_pop}
\end{Highlighting}
\end{Shaded}

\begin{verbatim}
## [1] 51
\end{verbatim}

\emph{3.Podemos realizar la misma operación que en el ejercicio anterior
usando la función which.min. Escribe una línea de código que haga esto.}

\begin{Shaded}
\begin{Highlighting}[]
\NormalTok{min\_ind\_pop2 }\OtherTok{=} \FunctionTok{which.min}\NormalTok{(pop)}
\NormalTok{min\_ind\_pop2}
\end{Highlighting}
\end{Shaded}

\begin{verbatim}
## [1] 51
\end{verbatim}

\begin{Shaded}
\begin{Highlighting}[]
\CommentTok{\#Vemos que coincide con el apartado anterior}
\end{Highlighting}
\end{Shaded}

\emph{4.Ahora sabemos cuán pequeño es el estado más pequeño y qué fila
lo representa. ¿Qué estado es? Define una variable states para que sea
los nombres de los estados del data frame murders. Reporta el nombre del
estado con la población más pequeña.}

\begin{Shaded}
\begin{Highlighting}[]
\CommentTok{\#Estado más pequeño:}
\NormalTok{murders}\SpecialCharTok{$}\NormalTok{state[min\_ind\_pop]}
\end{Highlighting}
\end{Shaded}

\begin{verbatim}
## [1] "Wyoming"
\end{verbatim}

\begin{Shaded}
\begin{Highlighting}[]
\CommentTok{\#Definimos variable states}
\NormalTok{states }\OtherTok{=}\NormalTok{ murders}\SpecialCharTok{$}\NormalTok{state}
\end{Highlighting}
\end{Shaded}

\emph{5.Puedes crear un data frame utilizando la función data.frame.
Utiliza la función rank para determinar el rango de población de cada
estado desde el menos poblado hasta el más poblado. Guarda estos rangos
en un objeto llamado ranks. Luego, crea un data frame con el nombre del
estado y su rango. Nombra el data frame my\_df.}

\begin{Shaded}
\begin{Highlighting}[]
\NormalTok{ranks }\OtherTok{=} \FunctionTok{rank}\NormalTok{(pop)}
\NormalTok{my\_df }\OtherTok{\textless{}{-}} \FunctionTok{data.frame}\NormalTok{(}\AttributeTok{states =}\NormalTok{ states , }\AttributeTok{rank =}\NormalTok{ ranks)}
\FunctionTok{head}\NormalTok{(my\_df)}
\end{Highlighting}
\end{Shaded}

\begin{verbatim}
##       states rank
## 1    Alabama   29
## 2     Alaska    5
## 3    Arizona   36
## 4   Arkansas   20
## 5 California   51
## 6   Colorado   30
\end{verbatim}

\emph{6.Repite el ejercicio anterior, pero esta vez ordena my\_df para
que los estados estén en orden de menos poblado a más poblado.
Sugerencia: cree un objeto ind que almacene los índices necesarios para
poner en orden los valores de la población. Luego, use el operador de
corchete {[} para reordenar cada columna en el data frame.}

\begin{Shaded}
\begin{Highlighting}[]
\NormalTok{ind }\OtherTok{=} \FunctionTok{order}\NormalTok{(pop)}
\NormalTok{my\_df2 }\OtherTok{\textless{}{-}}\NormalTok{ my\_df[ind,]}
\FunctionTok{head}\NormalTok{(my\_df2)}
\end{Highlighting}
\end{Shaded}

\begin{verbatim}
##                  states rank
## 51              Wyoming    1
## 9  District of Columbia    2
## 46              Vermont    3
## 35         North Dakota    4
## 2                Alaska    5
## 42         South Dakota    6
\end{verbatim}

\emph{7.El vector na\_example representa una serie de conteos. La
función is.na devuelve un vector lógico que nos dice qué entradas son
NA. Asigna este vector lógico a un objeto llamado ind y determina
cuántos NAs tiene na\_example.}

\begin{Shaded}
\begin{Highlighting}[]
\FunctionTok{data}\NormalTok{(}\StringTok{"na\_example"}\NormalTok{)}
\FunctionTok{str}\NormalTok{(na\_example)}
\end{Highlighting}
\end{Shaded}

\begin{verbatim}
##  int [1:1000] 2 1 3 2 1 3 1 4 3 2 ...
\end{verbatim}

\begin{Shaded}
\begin{Highlighting}[]
\NormalTok{ind }\OtherTok{=} \FunctionTok{is.na}\NormalTok{(na\_example)}
\CommentTok{\#total = sum(ind)}
\FunctionTok{length}\NormalTok{(}\FunctionTok{which}\NormalTok{(ind }\SpecialCharTok{==} \ConstantTok{TRUE}\NormalTok{))}
\end{Highlighting}
\end{Shaded}

\begin{verbatim}
## [1] 145
\end{verbatim}

\emph{8.Ahora calcula nuevamente el promedio, pero solo para las
entradas que no son NA. Sugerencia: recuerde el operador !}

\begin{Shaded}
\begin{Highlighting}[]
\NormalTok{ind\_new }\OtherTok{=}\NormalTok{ na\_example[ind}\SpecialCharTok{!=}\ConstantTok{TRUE}\NormalTok{]}
\FunctionTok{mean}\NormalTok{(ind\_new)}
\end{Highlighting}
\end{Shaded}

\begin{verbatim}
## [1] 2.301754
\end{verbatim}

\textbf{\emph{PARTE II:}}

\emph{1.Anteriormente, creamos este data frame:}

\begin{Shaded}
\begin{Highlighting}[]
\NormalTok{temp }\OtherTok{\textless{}{-}} \FunctionTok{c}\NormalTok{(}\DecValTok{35}\NormalTok{, }\DecValTok{88}\NormalTok{, }\DecValTok{42}\NormalTok{, }\DecValTok{84}\NormalTok{, }\DecValTok{81}\NormalTok{, }\DecValTok{30}\NormalTok{)}
\NormalTok{city }\OtherTok{\textless{}{-}} \FunctionTok{c}\NormalTok{(}\StringTok{"Beijing"}\NormalTok{, }\StringTok{"Lagos"}\NormalTok{, }\StringTok{"Paris"}\NormalTok{, }\StringTok{"Rio de Janeiro"}\NormalTok{,}\StringTok{"San Juan"}\NormalTok{, }\StringTok{"Toronto"}\NormalTok{)}
\NormalTok{city\_temps }\OtherTok{\textless{}{-}} \FunctionTok{data.frame}\NormalTok{(}\AttributeTok{name =}\NormalTok{ city, }\AttributeTok{temperature =}\NormalTok{ temp)}
\end{Highlighting}
\end{Shaded}

\emph{Vuelve a crear el data frame utilizando el código anterior, pero
agrega una línea que convierta la temperatura de Fahrenheit a Celsius.
La conversión es \(C=59×(F−32)\) .}

\begin{Shaded}
\begin{Highlighting}[]
\NormalTok{temp }\OtherTok{\textless{}{-}} \FunctionTok{c}\NormalTok{(}\DecValTok{35}\NormalTok{, }\DecValTok{88}\NormalTok{, }\DecValTok{42}\NormalTok{, }\DecValTok{84}\NormalTok{, }\DecValTok{81}\NormalTok{, }\DecValTok{30}\NormalTok{)}
\NormalTok{celsius }\OtherTok{\textless{}{-}} \DecValTok{5}\SpecialCharTok{/}\DecValTok{9} \SpecialCharTok{*}\NormalTok{ (temp}\DecValTok{{-}32}\NormalTok{)}
\NormalTok{city }\OtherTok{\textless{}{-}} \FunctionTok{c}\NormalTok{(}\StringTok{"Beijing"}\NormalTok{, }\StringTok{"Lagos"}\NormalTok{, }\StringTok{"Paris"}\NormalTok{, }\StringTok{"Rio de Janeiro"}\NormalTok{,}\StringTok{"San Juan"}\NormalTok{, }\StringTok{"Toronto"}\NormalTok{)}
\NormalTok{city\_temps }\OtherTok{\textless{}{-}} \FunctionTok{data.frame}\NormalTok{(}\AttributeTok{name =}\NormalTok{ city, }\AttributeTok{temperature =}\NormalTok{ celsius)}
\FunctionTok{head}\NormalTok{(city\_temps)}
\end{Highlighting}
\end{Shaded}

\begin{verbatim}
##             name temperature
## 1        Beijing    1.666667
## 2          Lagos   31.111111
## 3          Paris    5.555556
## 4 Rio de Janeiro   28.888889
## 5       San Juan   27.222222
## 6        Toronto   -1.111111
\end{verbatim}

\emph{2.¿Cuál es la siguiente suma \(1+1/2^2+1/3^2+…1/100^2\)? Ayuda:
gracias a Euler, sabemos que debería estar cerca de \(π^2/6\).}

\begin{Shaded}
\begin{Highlighting}[]
\CommentTok{\#Planteamos dos formas diferentes de resolver el problema.}
\CommentTok{\#Primera manera:}
\NormalTok{vec }\OtherTok{=} \FunctionTok{c}\NormalTok{(}\DecValTok{1}\SpecialCharTok{:}\DecValTok{100}\NormalTok{)}
\NormalTok{sum1 }\OtherTok{=} \FunctionTok{sum}\NormalTok{(}\DecValTok{1}\SpecialCharTok{/}\NormalTok{(vec}\SpecialCharTok{\^{}}\DecValTok{2}\NormalTok{))}
\NormalTok{sum1}
\end{Highlighting}
\end{Shaded}

\begin{verbatim}
## [1] 1.634984
\end{verbatim}

\begin{Shaded}
\begin{Highlighting}[]
\NormalTok{(pi}\SpecialCharTok{\^{}}\DecValTok{2}\NormalTok{)}\SpecialCharTok{/}\DecValTok{6}
\end{Highlighting}
\end{Shaded}

\begin{verbatim}
## [1] 1.644934
\end{verbatim}

\begin{Shaded}
\begin{Highlighting}[]
\CommentTok{\#Segunda manera:}
\NormalTok{sum2 }\OtherTok{=} \DecValTok{0}
\ControlFlowTok{for}\NormalTok{(i }\ControlFlowTok{in} \DecValTok{1}\SpecialCharTok{:}\DecValTok{100}\NormalTok{)\{}
\NormalTok{  sum2 }\OtherTok{=}\NormalTok{ sum2 }\SpecialCharTok{+} \DecValTok{1}\SpecialCharTok{/}\NormalTok{(i}\SpecialCharTok{\^{}}\DecValTok{2}\NormalTok{)}
\NormalTok{\}}
\NormalTok{sum2}
\end{Highlighting}
\end{Shaded}

\begin{verbatim}
## [1] 1.634984
\end{verbatim}

\begin{Shaded}
\begin{Highlighting}[]
\NormalTok{(pi}\SpecialCharTok{\^{}}\DecValTok{2}\NormalTok{)}\SpecialCharTok{/}\DecValTok{6}
\end{Highlighting}
\end{Shaded}

\begin{verbatim}
## [1] 1.644934
\end{verbatim}

\emph{3.Calcula la tasa de asesinatos por cada 100000 habitantes para
cada estado y almacénela en el objeto murder\_rate. Luego, calcula la
tasa promedio de asesinatos para EE.UU. con la función mean. ¿Cuánto
vale la media?}

\begin{Shaded}
\begin{Highlighting}[]
\NormalTok{murder\_rate }\OtherTok{\textless{}{-}}\NormalTok{ murders}\SpecialCharTok{$}\NormalTok{total }\SpecialCharTok{/}\NormalTok{ murders}\SpecialCharTok{$}\NormalTok{population }\SpecialCharTok{*} \DecValTok{100000}
\NormalTok{murder\_mean }\OtherTok{\textless{}{-}} \FunctionTok{mean}\NormalTok{(murder\_rate)}
\NormalTok{murder\_mean}
\end{Highlighting}
\end{Shaded}

\begin{verbatim}
## [1] 2.779125
\end{verbatim}

\end{document}
